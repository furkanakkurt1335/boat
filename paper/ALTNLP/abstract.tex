\begin{abstract}
The value of quality treebanks is steadily increasing due to the important role they play in the development of natural language processing tools.
The creation of such treebanks is enormously labour-intensive and time consuming.
Therefore, tools that support the annotation process are very valuable.
This is especially the case for morphologically rich languages.
The BoAT annotation tool was developed for annotating dependency relations, which was used to create the manually annotated Boun Treebank (UD\_Turkish-BOUN) of Turkish National Corpus (TNC).
The extensive annotation experience revealed several opportunities of improvement.
Combined with our desire to create a multi-user web based annotation tool, this led to the development of BoAT v2.
The main objectives of the new tool are to: (1) provide further support for creating valid and consistent annotations, (2) significantly improve the user experience of the annotator, and (3) develop an open source and easily deployable web-based annotation tool and an API to benefit the scientific and education community.
This paper discusses the requirements, design, implementation of BoAT v2 with annotation examples that highlight the new features.
\end{abstract}


