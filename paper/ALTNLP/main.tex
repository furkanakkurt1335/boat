\documentclass[
]{ceurart}

\usepackage[utf8]{inputenc}

\begin{document}

\copyrightyear{2022}
\copyrightclause{Copyright for this paper by its authors.
  Use permitted under Creative Commons License Attribution 4.0
  International (CC BY 4.0).}

\conference{ALTNLP2022: The International Conference on Agglutinative Language Technologies as a challenge of Natural Language Processing, June 6-8, 2022, Koper, Slovenia}

\title{Web Annotation Tool BoAT for Turkish dependency parsing}

\author[1]{Suzan Uskudarli}[%
email=suzan.uskudarli@boun.edu.tr
]

\author[1]{Salih Furkan Akkurt}[%
email=furkan.akkurt@boun.edu.tr
]

\address[1]{ Department of Computer Engineering, Bogazici University, Bebek, 34342, İstanbul, Turkey }

\begin{abstract}
Some abstract
\end{abstract}

\begin{keywords}
natural language processing \sep
linguistics \sep
annotation
\end{keywords}

\maketitle

\section{Introduction}

Annotation is a crucial part of Natural Language Processing.
Manually annotating treebanks is an intense activity undertaken by annotators.
Automated tools for dependency parsing try to ease the annotation work.
The efficiency of the work is further increased by the feedback and implementation cycle of the software.

The automation of this work is more helpful in some languages than others.
Turkish being an agglutinative language, the words are parsed further and further into lemmas.
Parsing dependencies in a language like this manually is hard work.
The automation tries to alleviate some of this work.

Listening to our annotators' experiences using the first version and other annotation tools gave us ideas to implement on the tool further and make their experiences healthier.
Feedback have been ample for the initial desktop version regarding the user's experience.
This led us to design a database and web-based application where interannotator agreement would be smoother and automated.
Some parts of the tool have been removed or altered, corresponding to the feedback.
The initial feedback and the tests done promise much for the future of the tool.

This paper presents this new web-based tool.
In the next section, the related work is touched upon.
In the third section, requirements and design procedure is detailed.
In the fourth, how the implementation proceeded is explained.
The fifth section explains what features the tool contains.
The sixth section contemplates what the future of the tool may entail.

\section{Related Work}

BoAT v1~\cite{turk-etal-2019-turkish}, the initial desktop application. 
This tool has been used to reannotate the entire BOUN Treebank.
We try to improve on this tool in this paper.\newline
brat: an open source project for text annotation.\cite{brat}\newline
There are third-party tools presented in the UD tools webpage such as UD Annotatrix\cite{tyers-etal:2018} and DgAnnotator\cite{dgannotator}.

\section{Requirements and Design}

Users' feedback and in-depth interviews have been used to create an SRS (Software Requirements Specification) document.
This feedback led us to believe a new web-based tool should be developed from scratch.
Focus of the annotator makes out to be the first requirement.
Annotators use the tools sometimes for hours on end.
Software should make the work easier by reducing distractions and automating work that can be automated.

One of the requirements we have gathered for not breaking the focus of the annotator is for the software to use only keyboard for all tasks.
BoAT is a keyboard-oriented application.
User should not have to use a mouse for any type of task.
Since users lose focus after some time naturally, automating the entries and checking errors are high priorities in this respect.
Autocomplete should be provided for parts of the annotation where it is known what values can be entered.

One essential feature requested by the users is using the screen space in a compact manner and removing any clutter that's not used much.
Considering this, the previous application's checkboxes have been converted into a select.

A search API for the annotation database of the treebanks is another requirement.
An annotator should be able to search the treebank they are a part of to see other annotations done for a specific sentence or go to a sentence's annotation page to start annotating.
Sometimes a user may not be sure of an annotation and would like to consult previous annotations for similar cases.
They should be able to search the database in a complex way, by case, features or direct text, etc.
An outsider also should be able to use the API to get the annotations of a particular treebank to use in their tasks in a systematic way.

\section{Implementation}

The annotation page resembles the initial standalone application.
It includes a dependency graph and an editable table, both of which in sync.
The dependency graph of the initial tool and other 2 graphs have been added to this tool.
The user has the choice to select a type of graph or none.
The other 2 graphs have been selected due to space considerations.
These graphs are horizontal and linear.
The table is responsible for changing the word lines of \textit{conllu} files.
A user is able to upload a \textit{conllu} file and start annotating.
The system is responsible for checking if a file uploaded obeys the format of \textit{conllu}.

Errors are checked and annotations validated according to the UD framework and the language provided.
The scripts used for validation are the ones already provided by the framework.\cite{UD-git}

Django\cite{django} and Django REST Framework\cite{drf} are used in this tool.
Django is the backbone of the tool, serving the website with all its configurations.
Django REST Framework is used for the API part.

PostgreSQL\cite{psql} is used to provide a database for the server-side applications of the software.
This database is responsible for serving the entire data needed to operate the tool, the search API, etc.

A Python library \textit{spaCy}\cite{spacy} is used to provide linear dependency graphs.
Another JavaScript-based linear dependency graph\cite{spyssalo} making use of brat\cite{brat-vis} is used to provide graphs as well.
We believe the preferences of annotators may vary and giving them options in different parts of the screen is important.

\section{Features}

One of the most game-changing features in this version is the ability to cross-check annotations by implementing a network for annotators where they can see the annotations done by other annotators and if they disagree on some parts, they can interact outside the tool.
This can be helpful and a learning experience for annotators.

Also one other thing the interannotator agreement allows us is the possibility to see some anomalies in the Turkish part of the UD framework.
For example, if a sentence were annotated a way by many annotators but the UD validation script were giving errors for it, this might indicate the UD validation were lacking in this respect of the Turkish language.
Some modifications might be necessary and there could be a case for a proposal of change.

\subsection{API}

By using our annotated treebank search API, other parties wanting to check out the capabilities of the tool can retrieve our annotated treebanks.
Also, some parties wanting to use the treebanks in their computation tasks can easily extract information from our treebanks by using the API.
This opens up many possibilities for the use of the tool beyond the annotator's UI experience.

\subsection{Demo}

A demo will be provided on our university's NLP web page where the annotation capabilities of the tool built can be tested.

\section{Discussion}

What could be done further?

\bibliography{main}

\end{document}
