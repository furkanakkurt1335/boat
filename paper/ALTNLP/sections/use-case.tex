\section{Annotation Procedure}
\label{sec:annotation}
This section describes how an annotation is done in practice.

An annotator selects a sentence from a treebank.
An annotation table appears with the sentence parsed according to the \ud\ format.
They fill the cells of the table.
An annotator can make use of dependency graphs, error cards and search.
Dependency graphs are visual cues for how lemmas are dependent upon one another.
Errors are helpful reminders compliant with \ud.
Also, different columns of the table can be toggled.
Other than all the columns for each of the 10 tags of an annotation in the \conllu\ format, there are extra columns for every valid feature.
They can also search for annotations, using various fields (\textit{text}, \feats, etc.) in the query.
This way, an annotator can ensure consistency and maintain focus.
When an annotation is done, they can select the annotation to be "Complete".
