\section{Implementation}
\label{sec:implementation}

The annoation tool is implemented using  Python~\cite{python}, the web application development framework \textit{Django}~\cite{django} with the API framework \textit{Django REST Framework} (DRF)~\cite{drf}. The webpages use Bootstrap~\cite{bootstrap}.
PostgreSQL~\cite{psql} is used as a database.
The models reflect the \ud\ format of sentences.
Most of the sentence annotation functionalities are similar to \boatvone.
User input is validated according to  \ud\ and errors are reported on the annotation page. 
Three alternative forms of dependency graph visualizations are supported, two of which are compact horizontal to reduce the required screen real estate~\cite{spacy,spyssalo,spyssalo}.


The following features have been implemented to support the creation of valid annotations with increased speed in a collaborative manner:


\begin{itemize}[before=\normalfont, font=\itshape, align=left,noitemsep,topsep=0pt,parsep=0pt,partopsep=0pt,labelwidth=5cm,align=left,itemindent=5cm]
    \item[Treebanks handling:]
        Users can work on multiple treebanks.
        The can upload new treebanks as \conllu\ files.
        Treebanks and their annotations are stored in a database.

    \item[Sentence annotation:]
    	An annotator selects the sentence they want to annotate. 
        The sentence annotation page is very similar to \boatvone.
        It consists of three main parts: (1) A table with rows for every lemma in the sentence and columns, which represent the \ud\ tags of a sentence (\form, \lemma, \deprel, etc.), corresponding to their annotations (2) the dependency graph of the sentence; and (3) errors from validation according to the \ud\ framework.
        The dependency graph and errors are synchronized with the annotations.
        Several dependency graph presentations are supported to suit the annotator preference.
        Vertical graphs can consume a significant amount of screen real-estate, which can lead to loss of focus in long sentences that are common in agglutinative languages.

        An annotator may need to stop the annotation of a sentence for some reason (i.e. complexity or external interruption).
        To capture the state of an annotation, a status is introduced for this purpose, which may take on on of these values: ``New", ``Draft" and ``Complete". 
        An annotation that has not been annotated has the statys ``new''. 
	An annotator can set the status to ``Draft'' or ``Complete''. 
        %The status of an annotation is also shown in the search results.

        The annotator is able to perform almost all operations via keyboard action, based on the demand of the annotators (see Section~\ref{sec:requirements} for details).
        Upon the experience of annotating a Turkish treebank, the annotators have reported being very pleased with the convenience and speed resulting from keyboard-based interaction.

    \item[Improved searching for reference and consistency:]
        Users are able to search for previously annotated sentences in combinations of surface text, tags, and features. 
		Without a good search feature an annotator would have to manually search the \conllu\ file for relevant cases (i.e. how to annotate some \form's \upos\ tag).
		The \form\ might have been inconsistently annotated, alas it is unlikely that manual searched would reveal this case. 
		In such a case the annotator would likely use the first encountered as a reference. 

        The situation gets more complicated for syncretic morphemes such as \textit{-ki} in Turkish.
        For example, in the sentence ``Evdeki halılar yıkandı.'' (\textit{The rugs at home were washed.}), the \textit{-ki} acts as an adjectivizer.
        However, in ``Benim halılarım yün, Ayşeninkiler sentetik.'' (\textit{My rugs are woolen. Ayşe's are synthetic.}), it is pronominal.
        Searching for sentences where the \textit{-ki} morpheme occurs via text search would be hopeless with as there would be too many hits since they occur very frequently.

        To facilitate effective searching, we have implemented search functionality based on combinations of text and \ud\ tags.
        Regular expression based search is also supported.
        This feature supports annotators to share experiences, which consequently is expected to result in more accurate and consistent treebanks.
        
    \item[Inter-annotator agreement:]
        The consistency of annotations among annotators is an indicator of quality of the resulting resource.
        Inter-annotator agreement to computes the consistency among annotators.
        Since this tool keeps track of annotator actions, such computations are straightforward.

\end{itemize}
