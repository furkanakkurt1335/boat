\section{Implementation}
\label{sec:implementation}

The Django\cite{django} Web application development framework and its Django REST Framework\cite{drf} were used to develop our annotation tool.
Django REST Framework were used for the API part.
% API
% backend database
% changed models for wordlines, helping API

Errors are checked and annotations validated according to the UD framework and the language provided.
The scripts used for validation are the latest ones already provided by the framework.\cite{UD-git}

PostgreSQL\cite{psql} is used to provide a database for the server-side applications of the software.
This database is responsible for serving the entire data needed to operate the tool, the search API, etc.

A Python library \textit{spaCy}\cite{spacy} is used to provide linear dependency graphs.
Another JavaScript-based linear dependency graph\cite{spyssalo} making use of brat\cite{brat-vis} is used to provide graphs as well.
The preferences of annotators may vary and giving them options in different parts of the screen is important.

\subsection{API}
\label{sec:api}

By using our annotated treebank search API, other parties wanting to check out the capabilities of the tool can retrieve our annotated treebanks.
Also, some parties wanting to use the treebanks in their computation tasks can easily extract information from our treebanks by using the API.
This o

% Docker


pens up many possibilities for the use of the tool beyond the annotator's UI experience.

\subsection{Demo}
\label{sec:demo}

A demo will be provided on our university's NLP web page where the annotation capabilities of the tool built can be tested.
