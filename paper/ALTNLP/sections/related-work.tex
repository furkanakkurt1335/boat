\section{Related Work}
\label{sec:related}

There are various tools for creating dependency annotations such as \textit{brat}~\cite{brat}, \textit{UD Annotatrix}~\cite{tyers-etal:2018} and \textit{DgAnnotator}~\cite{dgannotator}
These tools mostly rely on mouse-based user interaction, which tend to be inefficient for annotating MRLs.

There are also tools developed specifically for Turkish.
The ITU Treebank Annotation Tool~\cite{pamay-etal-2015-annotation} is a tool written in Java that has several versions.
It offers semi-automated support for annotators through analyzers for creating new datasets as well as correcting the existing Turkish treebanks.
It mostly relies on mouse-based interactions and doesn't support \ud. 

\boatvone~\cite{turk-etal-2019-turkish} is a standalone application that supports keyboard-based input while displaying the annotations as a table as well as a dependency graph.
It also supports UD~\cite{UD}. 
This tool was used to create the \bountreebank~\cite{trk2020resources,UD-Boun-Treebank} -- a manually annotated Turkish dependency treebank comprising 9,761 sentences. It also uses analyzers to process sentences. 
The feedback received from this effort indicates that the keyboard based input is much more efficient and takes less time for creating quality treebanks.

The work presented in this paper is based on the experiences from \boatvone\ with continuous feedback from an annotator. 
Like many tools, it supports \ud, and it also supports keyword-based input to allow the annotation process to be performed without breaking the flow. 
Furthermore, the tool is developed as a web application with an API to support ease of deployment and multi-annotator scenarios. 
