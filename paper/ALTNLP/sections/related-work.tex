\section{Related Work}
\label{sec:related}

Annotation tools may be characterized in terms of the support they provide for languages, user interface modalities, standards, and mutiple annotators. 
Adherence to standards is recommended to get the most benefit from the annotated data sources. 
The Universal Dependencies~\ref{ud} is the most widely accepted standard for dependency annotations. 
It is an actively growing standard that intends to cover  all languages. 
Support for agglutinative languages is evolving.


Several dependency annotation tools have been proposed such as \textit{brat}~\cite{brat}, \textit{UD Annotatrix} and \textit{DgAnnotator}~\cite{UD}.
These tools mostly rely on mouse-based user interaction, which are inefficient for annotating agglutinative languages due to the need for entering numerous features for  most \form s.

The ITU Treebank Annotation Tool~\cite{pamay-etal-2015-annotation} is developed for Turkish. 
It is a written in Java as a stand-alone tool and has several versions.
It offers semi-automated support for annotators through analyzers for creating new datasets as well as correcting the existing Turkish treebanks.
It mostly relies on mouse-based interactions and doesn't support the \ud\ framework. 



\boatvone~\cite{turk-etal-2019-turkish} is a standalone annotation tool for treebanks compatible with the \ud\ framework~\cite{UD} and supports keyboard-based input while displaying the annotations as a table with a corresponding dependency graph for visualization.
Although mouse input is supported for most tasks, using only keyboard is preferred as our experiences with the tool show that it tends to be more efficient.
This tool was used to create the \bountreebank~\cite{turk2021resources,UD-Boun-Treebank} -- a manually annotated Turkish dependency treebank comprising 9,761 sentences.
It also uses analyzers to process treebanks with the \conllu\ format of \ud. 
The feedback received from this work indicates that the keyboard based input is much more efficient and takes less time for creating quality treebanks.

WebAnno~\cite{webanno} which supports multiple users and is also web-based, similar to \boatvtwo and it relies on mouse interactions.
To annotate features of a \form, it requires many clicks which makes it impractical, especially for MRLs.
It does not have a dependency graph for visualization of an annotation, unlike \boatvone. Instead of a dependency graph, it shows many sentences simultaneously, which distracts the annotator.

The work presented in this paper is based on the experiences from \boatvone\ with continuous feedback from an annotator. 
Like its predecessor and many other tools, it supports \ud, and it also supports keyword-based input to allow the annotation process to be performed without breaking the flow. 
Furthermore, the tool is developed as a web application with an API to support ease of deployment and multi-annotator scenarios. 
Feedback indicated that big treebanks like the \bountreebank\ require many annotators collaborating and treebanks of that size are only increasing with time.
Accordingly, supporting a multi-user experience of collaboration is a high priority of the new tool.
