\section{Related Work}
\label{sec:related}

Annotation tools may be characterized in terms of the support they provide for languages, user interface modalities, standards, and multiple annotators.
Adherence to standards is recommended to get the most benefit from the annotated data sources.
Universal Dependencies~\cite{UD} is the most widely accepted standard for dependency annotations.
It is an actively growing standard that intends to cover all languages and its support for agglutinative languages is evolving.

Several dependency annotation tools have been proposed such as \textit{brat}~\cite{brat}, \textit{UD Annotatrix}~\cite{tyers-etal:2018} and \textit{DgAnnotator}~\cite{dgannotator,UD-tools}.
These tools mostly rely on mouse-based user interaction, which is inefficient for annotating agglutinative languages due to the need for extensive annotation for most tokens.

The ITU Treebank Annotation Tool~\cite{pamay-etal-2015-annotation} was developed for Turkish.
It was written in Java as an open-source standalone tool and has several versions.
It offers semi-automated support for annotators through analyzers for creating new datasets as well as correcting already existing Turkish treebanks.
It mostly relies on mouse-based interactions and doesn't support the \ud\ framework.

WebAnno~\cite{webanno} is a web-based annotation tool with multi-user support.
To annotate features of a token, it requires several mouse clicks which is impractical for MRLs.
It does not display a dependency graph, unlike \boatvone\ and \boatvtwo.
It presents several sentences simultaneously, which can be distracting.

\boatvtwo\ is the second iteration of \boatvone.
It improves on it by including a search functionality, a database to represent sentences in a more essential way rather than a plaintext file and an accessible web interface with an API for flexibility.
