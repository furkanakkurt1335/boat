\section{Discussion}
\label{sec:discussion}

We wanted a tool capable of the functionalities in \boatvone\ and also offer a network for annotators to collaborate on the same treebank.
We decided on a web application with an API for this purpose.
For the web framework, Django and extensible search capabilities of DRF served our purpose well.

Doğal Dil İşleme Platformu ("Natural Language Processing Platform") of Bogazici University's Computer Engineering Department have various NLP tools.
The tools mostly output results based on input without a user interface.
This is the first tool that has a UI in the platform and accordingly we needed to take some action regarding the integration of this novel tool.
Dockerizing the application will allow us to easily integrate it into the DIP platform of the department~\cite{DIP}.

Interviews with and feedback from annotators that had used \boatvone\ through the annotation of the \bountreebank{} and also had used other tools have been very helpful.
They shaped every part of the tool, from designing the API to the annotation view in the UI part.
It is an extensible tool and can be tailored for other use cases.
