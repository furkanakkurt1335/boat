\section{Discussion and Conclusions}
\label{sec:discussion}

\boatvtwo\ aimed to  extend the functionality of \boatvone\ as a collaborative web-based application to support annotation process based on previous experiences. 
We developed a web application that supports agglutinative languages as described in Section~\ref{sec:implementation}. 

The implementation choices served our goals well. 
We believe that having experts in linguistics and experienced annotators in agglutinative treebank creation was instrumental in understanding the requirements and design process. 
We held numerous elicitation interviews and further meetings for clarifications and feedback requests.

We used modern software development tools and management practices during the development lifecycle of this tool.
The development of an API enables various extensions of this tool and access to the treebanks.  
The containerization with Docker~\cite{docker} has facilitated easy delivery and deployment.
It will be made available on Boğaziçi University's NLP platform DIP~\cite{DIP} as a demo as well as an open source resource.
The availability of the tool as a user as well as developer is valuable for future use and developments. 

This tool is in testing phase and has had encouraging early feedback.
We are encouraged by the early responses to this tool and anticipate its extensions.
We plan to conduct further tests with the linguistics department. 
We also see potentials for customizing this tool to support annotation of rare languages.
