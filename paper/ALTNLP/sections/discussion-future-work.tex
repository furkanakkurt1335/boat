\section{Discussion and Conclusions}
\label{sec:discussion}

\boatvtwo\ aims to extend the functionality of \boatvone\ as a collaborative web-based application to support the annotation process based on previous experiences.
We developed a web application that supports agglutinative languages as described in Section~\ref{sec:implementation}.

The implementation choices served our goals well.
We believe that having experts in linguistics and experienced annotators in agglutinative treebank creation was instrumental in understanding the requirements and the design process.
We held numerous elicitation interviews and further meetings for clarifications and feedback requests.

We used modern software development tools and management practices during the development lifecycle of this tool.
The development of an API enables various extensions of this tool and access to the treebanks.
The containerization with Docker~\cite{docker} has facilitated easy delivery and deployment.
It will be made available on Boğaziçi University's NLP platform DIP~\cite{DIP} as a demo as well as an open-source resource.
The availability of the tool as a user as well as a developer is valuable for future use and developments.

This tool is in the testing phase and has had encouraging early feedback.
We compared the annotation of several sentences with the same number of tokens using \boatvone\ and \boatvtwo.
While we kept the number of words in a sentence constant, we did not use the same sentences since having previously annotated a sentence would impact the annotation on another version.
Keeping the number of words the same provides a somewhat comparable experience.
There was a noticeable speedup using \boatvtwo.
To annotate sentences with an average of 13 tokens took approximately 5.3 minutes using \boatvtwo\ in comparison to 8 minutes with \boatvone.
Among the new features that are most appreciated are autocompletion, condensed dependency tree representation, significant reduction on scrolling, keyword search, and search by morphological features.
The non-search related features are instrumental to retaining focus.

We are encouraged by the early responses to this tool and anticipate its extensions.
In fact, our implementation of \boatvtwo\ resulted in a revision request for \boatvone\ to include the focus enhancing features.
This also resulted in significant speedup and improved user experience, which was reported as a qualitative observation by an annotator.
We also see potentials for customizing this tool to support annotation of rare languages.

Presently, our testing is focused on more extensive cases (annotation of sentences with varying degrees of complexity) and, importantly, the multi-user functionalities.
For this purpose, we are in the process of recruiting several annotators with a background in linguistics.
