\section{Features}
\label{sec:features}

There are many new features and improvements in this tool, as well as the functionality that already existed in BoAT v1.
A user is able to upload a \textit{CoNLL-U} file and start annotating.
The file is parsed and checked for its format. It's rejected if incorrectly formatted, then uploaded to the database.
This way, other annotators working for the same treebank don't have to provide the same file.

The annotation page is very similar to BoAT v1.
It includes a dependency graph and an editable table, both of which in sync.
The dependency graph of the initial tool and other 2 graphs have been added to this tool.
The user has the choice to select a type of graph or none.
The other 2 graphs, which are both horizontal and linear, have been selected due to space considerations.
The annotation table is for editing the word lines of \textit{CoNLL-U} files.

% consistent annotation
% an annotator can struggle with a specific one; search helps
% busra'dan ornek; kisa (ya da uzun gerekiyorsa); istendi, 3 ornek geldi
An important feature in this version is the ability to cross-check annotations by implementing a network for annotators where they can see the annotations done by other annotators and if they disagree on some parts, they can interact outside the tool.
This can be helpful and a learning experience for annotators.
For an actual example, the annotator, responsible for the BOUN Treebank's annotation, encountered two different ways nouns of Zodiac signs were annotated.
For example, if in the sentence "Terazi rahatına düşkündür." (\textit{"Libras like comfort."}), "Terazi" was annotated

Also one other thing the interannotator agreement allows us is the possibility to see some anomalies in the Turkish part of the UD framework.
For example, if a sentence were annotated a way by many annotators but the UD validation script were finding it invalid, this might indicate the UD validation were lacking in this respect of the Turkish language.
Some modifications might be necessary and there could be a case for a proposal of change.
