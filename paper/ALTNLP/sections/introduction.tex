\section{Introduction}
\label{sec:introduction}
Treebanks are useful for various natural language processing tasks.
Manually annotating treebanks is an intense activity undertaken by annotators.
Several tools to support dependency parsing annotations have been developed.\cite{brat}\cite{dgannotator}
For morphologically-rich languages like Turkish, dependency annotation tends to be much more challenging.
Turkish being an agglutinative language, words are split into further lemmas.
Parsing dependencies in such a language manually is hard work.
Tools developed with drag-drop functionalities tend to disrupt the flow state of the annotator by requiring to use a mouse in the middle of a keyboard-intensive activity.
\boat{} v1, the tool developed and used for the annotation of the entire \bountreebank{}, had significant improvements on efficiency by making it a priority to be a keyboard-oriented application.

Annotator's experiences using the first version and other annotation tools have been taken into consideration to improve the tool further and make their experiences healthier.
Feedback have been ample for the standalone version regarding the user's experience.
Accordingly, we wanted to design a web-based application with a database and APIs where interannotator agreement would be smoother and automated.
By making it a web app, we increased the accessibility of the software so that it can be used for other NLP tasks or educational purposes.
We plan to dockerize it and make the API accessible to 3rd parties.
The initial feedback and the tests done promise much for the future of the tool.

This paper presents this new web-based tool.
In Section~\ref{sec:related}, the related work is touched upon.
In Section~\ref{sec:requirements}, elicited and specified software requirements and design procedure is detailed.
In Section~\ref{sec:implementation}, how the implementation proceeded is explained.
Section~\ref{sec:features} explains what features the tool contains.
In Section~\ref{sec:annotation}, a use case going through an annotation of a sentence is included.
Section~\ref{sec:discussion} contemplates what the future of the tool may entail.
Section~\ref{sec:conclusion} adds the final conclusions.
