\section{Discussion and Conclusions}
\label{sec:discussion}

\boatvtwo\ aims to extend the functionality of \boatvone\ as a collaborative web-based application to support the annotation process based on previous experiences.
We developed a web application that supports agglutinative languages as described in Section~\ref{sec:implementation}.

The implementation choices served our goals well.
We believe that having experts in linguistics and experienced annotators in agglutinative treebank creation was instrumental in understanding the requirements and the design process.
We held numerous elicitation interviews and further meetings for clarifications and feedback requests.

We used modern software development tools and management practices during the development lifecycle of this tool.
The development of an API enables various extensions of this tool and access to the treebanks.
The containerization with Docker~\cite{docker} has facilitated easy delivery and deployment.
It is available on Boğaziçi University's NLP platform~\cite{TULAP} as an open-source resource.
The availability of the tool as a user as well as a developer is valuable for future use and developments.

We are encouraged by the evaluation performed on this tool and anticipate its extensions.
In fact, our implementation of \boatvtwo\ resulted in a revision request for \boatvone\ to include the focus enhancing features.
This resulted in significant speedup and improved annotator experience, which was reported as a qualitative observation by an experienced annotator.

The feedback coming from the evaluation presents many ways that the tool can be improved.
For the tool's future, we plan on maintaining the current functionality, implementing annotation-centric new features, such as automated annotation of certain tags based on rules input by annotators of a treebank, or creating notes on annotations to communicate with other annotators through the application.
We also plan to implement support for various languages regarding autocompletion and annotation validation.
% annotation outside dep ann?