\begin{abstract}

% need to revise for journal
% user tests, evaluation

Value of quality treebanks is steadily increasing due to the crucial role they play in development of natural language processing tools.
Creation of such treebanks is enormously labor-intensive and time-consuming.
Especially when sizes of current treebanks are considered, tools that support the annotation process are essential.
Various annotation tools have been proposed; however, they are often not suitable for agglutinative languages such as Turkish.
The first version of Boğaziçi University Annotation Tool (\boatvone) was developed for annotating grammar and was subsequently used to create the manually annotated \bountreebank\ (UD\_Turkish-BOUN).
In this work, we report on the design and implementation of a web-based grammatical annotation tool (\boatvtwo) based on the experiences gained from the use of \boatvone, which revealed several opportunities of improvement.
\boatvtwo\ is a multi-user and web-based grammatical annotation tool, designed with a focus on the annotator user experience to yield valid and consistent annotations.
The main objectives of the tool are to: (1) support creating valid and consistent annotations with increased speed, (2) significantly improve the user experience of annotators, (3) support collaboration among annotators, and (4) provide an open-source and easily deployable web-based annotation tool with a flexible application programming interface (API) to benefit the scientific community.
This paper discusses the requirements elicitation, design, and implementation of \boatvtwo\ along with examples.

\keywords{natural language processing, linguistic annotation, annotation tool, web application, dependency parsing, Universal Dependencies}

\end{abstract}
