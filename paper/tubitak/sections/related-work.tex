\section{Related Work}
\label{sec:related}

Annotation tools may be characterized in terms of their accessibility and the support they provide for different languages, various annotation categories, user interface modalities, annotation related standards, and multiple annotators.
Adherence to standards is recommended to get the most benefit from the annotated data sources.
Universal Dependencies~\cite{UD} is an actively growing standard for annotation of treebanks that intends to cover all human languages and its support for agglutinative languages is evolving.

Several dependency annotation tools have been proposed with many of them being UD-compliant, such as \textit{brat}~\cite{brat}, \textit{UD Annotatrix}~\cite{ud-annotatrix} and \textit{DgAnnotator}~\cite{dgannotator}.
These tools are not developed for a specific language and can be used with a variety of languages.
They mostly rely on mouse-based user interactions, which is inefficient for annotation of agglutinative languages due to the need for extensive annotation for most tokens.
They are able to produce UD-compliant treebanks.

\textbf{brat} is a web-based \textit{text} annotation tool which provides a collaborative annotation environment. As such, it is not restricted to dependency annotation.

\textbf{UD-Annotatrix} aims to be a simpler and cleaner dependency annotation tool than brat.

\textbf{The Dependency Grammar Annotator} (DGA or DgAnnotator) is another dependency annotation tool, written in Java. It enables its users to annotate POS and dependency relation tags via mouse clicks.

\textbf{ITU Treebank Annotation Tool}~\cite{itu-annotation-tool} was developed for Turkish.
It was written in Java as an open-source standalone tool and has several versions.
It has three stages of annotation: morphological analysis, morphological disambiguation and syntax analysis.
It offers semi-automated support for annotators through analyzers for creating new datasets as well as correcting already existing Turkish treebanks.
It mostly relies on mouse-based interactions and doesn't support the \ud\ framework.
This tool has been used to annotate the ITU Web Treebank~\cite{itu-web-tb}.

\textbf{WebAnno}~\cite{webanno} is a web-based open-source annotation tool, that is not restricted to dependency annotations but has support for morphological, syntactical, and semantic annotations also, with multi-user support.
To annotate features of a token, it requires several mouse clicks which is impractical for MRLs.
The sentences to be annotated are displayed vertically in a single screen, unlike many tools that focus on a single sentence at a time.

% olcay paper
\textbf{StarDust}~\cite{stardust} is a standalone dependency annotation tool.
It uses 2 layers: morphological disambiguation and dependency annotation layers, sequentially.
In the dependency annotation layer, annotators are expected to click on a dependeny word and drag the cursor to the head word.
It stores each sentence's annotation in a different file.
For backups of the annotations, it uses Dropbox~\cite{dropbox}, which is a proprietary cloud storage platform.
There were many treebanks annotated with this tool: 5 Turkish and 1 English treebank.

\textbf{BoAT-v2} is the second iteration of \boatvone.
It improves on it by including a search functionality, a database to represent sentences in a more essential way rather than a plaintext file and an accessible web interface with an API for flexibility.
\boatvtwo\ has been developed to reduce clutter that \boatvone\ was found to have in parts of the screen by the feedback coming from the previous \boatvone\ usage.

% summary table, if possible: this exists, this not
\begin{table}[ht!]
    \label{table:rel-work}
    \centering
    \begin{tabular}{|p{3cm}|p{3cm}|p{3cm}|p{3cm}|p{3cm}|}
        \hline
        \multicolumn{5}{|c|}{\textbf{Comparison of Annotation Tools}} \\ \hline\hline
        \textbf{Tool} & \textbf{Multi-language} & \textbf{Modality} & \textbf{Standards} & \textbf{Multi-annotator} \\\hline
        brat &  &  &  &  \\\hline
        UD Annotatrix &  &  &  &  \\\hline
        DgAnnotator &  &  &  &  \\\hline
        ITU &  &  &  &  \\\hline
        WebAnno &  &  &  &  \\\hline
        StarDust &  &  &  &  \\\hline
        \boatvtwo\ & \checkmark & Mouse \& keyboard & UD & \checkmark \\\hline
    \end{tabular}
    \caption{Properties of various annotation tools}
\end{table}

% need to detail a lot
