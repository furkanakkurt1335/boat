\section{Implementation}
\label{sec:implementation}

The annotation tool is implemented using Python~\cite{python}, the web application development framework \textit{Django}~\cite{django} and the API framework \textit{Django REST Framework} (DRF)~\cite{drf}. The webpages use Bootstrap~\cite{bootstrap}.
PostgreSQL~\cite{psql} is used as a database management system.
The models reflect the \ud\ format of sentences.
Most of the sentence annotation functionalities are similar to \boatvone.
User input is validated according to \ud\ and errors are reported on the annotation page.
Three alternative forms of dependency graph visualizations are supported, two of which are newly added, compact and horizontal to reduce the required screen real estate~\cite{spacy,spyssalo}.

The following features have been implemented to support the creation of valid annotations with increased speed in a collaborative manner:
\begin{itemize}[before=\normalfont, font=\itshape, align=left,noitemsep,topsep=0pt,parsep=3pt,partopsep=0pt,labelsep=3pt,align=left]
    \item[Treebank handling:]
        The tool should support the annotation of multiple treebanks.
        \boatvtwo\ uses a database to persist the annotations of multiple annotators of multiple treebanks.

    \item[Sentence annotation:]
    	An annotator selects the sentence they want to annotate.
        The sentence annotation page is very similar to \boatvone.
        It consists of three main parts: (1) A table with rows for every token in a sentence and columns, which represent the \ud\ tags of a sentence, corresponding to their annotations; (2) the dependency graph of the sentence; and (3) errors from validation according to the \ud\ framework.
        The dependency graph and errors are synchronized with the annotations.
        Several dependency graph presentations are supported to suit the annotator's preference.
        2 horizontal dependency graphs are supported since vertical graphs can consume a significant amount of screen real estate, which can lead to loss of focus in long sentences common in agglutinative languages.
        One vertical dependency graph resembling a tree is also supported in case the annotator prefers it. This vertical graph was also supported in \boatvone.

        An annotator may need to stop the annotation of a sentence for some reason (i.e. complexity or external interruption).
        To capture the state of an annotation, status is introduced in \boatvtwo\ with values of ``New", ``Draft" and ``Complete".
        The status of a sentence that has not been annotated is ``New''.
	    An annotator can set the status to ``Draft'' or ``Complete''.

        The annotator is able to perform almost all operations, more than what \boatvone\ allowed, via keyboard action, based on the demand of the annotators (see Section~\ref{sec:requirements} for details).
        Upon the experience of annotating a Turkish treebank with \boatvone, the annotators have reported being very pleased with the convenience and speed resulting from keyboard-based interaction.

    \item[Searching for reference and consistency:]
        A search functionality is introduced in \boatvtwo.
        With it, users are able to search for previously annotated sentences based on combinations of surface text, \ud\ tags, and individual morphological features.
		Without a good search feature, an annotator would have to manually search their \conllu\ files for relevant cases (e.g. how to annotate a certain surface form's \upos\ tag).
		The form might have been inconsistently annotated, alas it is unlikely that a manual search would reveal this problem.
		In such a case, the annotator would likely use the first encountered usage as a reference.
        The situation gets more complicated for syncretic morphemes such as \textit{-ki} in Turkish.
        For example, in the sentence ``Evdeki halılar yıkandı.'' (\textit{The rugs at home were washed.}), the \textit{-ki} acts as an adjectivizer.
        However, in ``Benim halılarım yün, Ayşeninkiler sentetik.'' (\textit{My rugs are woolen. Ayşe's are synthetic.}), it is pronominal.
        Searching for sentences where the \textit{-ki} morpheme occurs via basic text search of a file would be hopeless as there would be too many hits since they occur very frequently, owing to its part in the grammar and brevity as a word.

        To facilitate effective searching, we have implemented search functionality based on combinations of text and \ud\ tags.
        Regular expression-based search is also supported.
        This feature supports annotators to share experiences, which consequently is expected to result in more accurate and consistent treebanks.

    \item[Inter-annotator agreement:]
        The consistency of annotations among annotators is an indicator of the quality of the resulting resource.
        Inter-annotator agreement computes the consistency among annotators.
        Since this tool keeps track of annotator actions, unlike its predecessor, such computations are straightforward.
        Some visualizations shall accompany these statistics.

\end{itemize}
