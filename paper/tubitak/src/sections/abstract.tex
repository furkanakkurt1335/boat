\begin{abstract}
% abstract limit is 300 words ; currently it's 300 exact, 1/8/2023 23:58.
The significance of high-quality treebanks has been steadily increasing due to their crucial role in developing natural language processing tools.
The creation of such treebanks is an enormously labor-intensive and time-consuming task due to the need for qualified annotators and the quantity of data to be annotated.
Tools that facilitate the annotation process have been developed, however, their interfaces are usually designed for analytic languages.
Agglutinative languages tend to be richer in their morphology and require extensive annotations that significantly differ from annotations in analytic languages.
Most of these tools are not well-suited for annotations in agglutinative languages such as Turkish.
% Thus, lemmas in these languages bear more features on average when compared to analytic languages.
% Moreover, the morphological typology of a language also affects the word order, often by allowing freer word order or less restricted scrambling and topicalization.
% word order ; extensive annotation ; lang these depend on compared to
% many tools developed for analytic langs
% Hence, many agglutinative languages also have longer dependency arc lengths.
As such, low-resource agglutinative languages need suitable tools to accelerate the creation of high-quality data resources.
% As a result, the need for an annotation tool that appropriately supports the annotator while helping them focus is very prominent.
This paper describes the design and implementation of a web-based collaborative grammatical annotation tool designed to expedite and facilitate the annotation of large treebanks, especially in agglutinative languages.
This work builds on extensive experience with another grammatical annotation tool developed previously to annotate a Turkish treebank.
The main objectives of this tool are to: (1) support the creation of valid and consistent annotations, (2) increase the speed of the annotation process, (3) improve the user experience of annotators, (4) support collaboration among annotators, and (5) provide an easily deployable and open-source web-based annotation tool with an application programming interface (API) to benefit the scientific community.
This paper discusses the requirements elicitation, design, implementation, and evaluation of this tool along with examples.

\keywords{Natural language processing, Linguistic annotation, Annotation tool, Web application, Dependency parsing, Universal dependencies}

\end{abstract}
