\section{Related Work}
\label{sec:related}

Annotation tools can be categorized in terms of their accessibility, and the support they provide for different languages, various annotation frameworks, user interface modalities, annotation-related standards, multiple annotators, searching of annotations, and automation.
Adherence to standards is recommended to get the most benefit from the annotated language resources.
Universal Dependencies~\cite{ud} (\ud) is an actively growing standard for the annotation of treebanks that intends to cover all languages and its support for agglutinative languages is evolving.

Numerous dependency annotation tools have been developed with many of them being UD-compliant, such as brat~\cite{brat}, UD Annotatrix~\cite{ud-annotatrix}, and DGA~\cite{dgannotator}.
Most of these tools are capable of annotating a variety of languages and all the tools that are detailed below do so.
Most tools require mouse-based user interactions, which is inefficient for the annotation of agglutinative languages due to several reasons, including the need for extensive annotation for most tokens.

% 2011
ITU Treebank Annotation Tool~\cite{itu-annotation-tool} is an open-source standalone tool with 3 versions.
It was developed specifically for Turkish.
It has three stages of annotation: morphological analysis, morphological disambiguation, and syntax analysis.
It offers semi-automated support for annotators through analyzers for creating new datasets as well as correcting already existing Turkish treebanks.
It mostly relies on mouse-based interactions and doesn't have support for the \ud\ framework.
This tool has been used to annotate the ITU Web Treebank~\cite{itu-annotation-tool}. % same ref with tool
Its latest version (3) was written in Java, requiring Java to be installed on the system used for annotations.

% 2012
brat~\cite{brat} is a web-based \textit{text} annotation tool which provides a collaborative annotation environment.
Being web-based, it's usable across different platforms.
It's very flexible in its annotation types and has support for many annotation tasks such as dependency, entity, or coreference annotation.
This tool is also used for the visualization of annotations.
Annotators use a mouse and keyboard together to annotate.
Annotators are able to transfer their annotations in a file format called \textit{standoff} specific to brat.
It is configurable to support annotations compliant with \ud\ and is used for visualization in \ud's own documentation system.
brat supports real-time collaboration, even allowing annotators to edit the same annotation.
It supports searching the documents or their annotations through using various filters.

% 2015
The Dependency Grammar Annotator~\cite{dgannotator} (also known as DGA or DgAnnotator) is a dependency annotation tool.
It enables its users to annotate POS and dependency relation tags via mouse clicks.
It supports the transfer of annotations in UD-compliant formats.
It doesn't support multiple annotators to work on the same treebank.
There is no search functionality.
It was written in Java~\cite{java} and consequently requires Java.

% 2016
WebAnno~\cite{webanno} is a web-based open-source annotation tool, that is not restricted to dependency annotations but has support for morphological, syntactical, and semantic annotations also, with multi-user support.
To annotate the morphological features of a token, it requires several mouse clicks which is impractical for MRLs.
The sentences to be annotated are displayed vertically on a single screen, unlike many tools that focus on a single sentence at a time.
It supports the curation of annotations created by multiple annotators.
It has two installment options, standalone and server.
The standalone version requires Java, while the server version requires a Tomcat~\cite{tomcat} server with a MySQL~\cite{mysql} database.
% may want to add inception as improvement over this

% 2018
UD-Annotatrix~\cite{ud-annotatrix} is a \ud-specific dependency annotation tool that aims to be simpler than brat.
Mouse usage is necessary, in addition to keyboard entries, for annotation.
It also supports real-time collaboration if the tool is deployed in a server.
It employs annotation standards of \ud\ and CG3~\cite{cg3}.
It doesn't have a search functionality.

% 2019
CoNLL-U Editor~\cite{conll-u_editor} is a web-based annotation tool, specific to the \conllu~\cite{conll} format of \ud.
It saves every edit of annotations by version control.
If specific lists of part-of-speech tags are provided by the annotators, it supports auto-completion during annotation.
Multiple annotators are able to annotate the same treebank if they don't work on the same sentence at a time.
Annotators can search using any combination of linguistic tags of \ud\ using the tool.
It requires a Java-based server.

% Olcay's paper, 2022
StarDust~\cite{stardust} is a standalone dependency annotation tool.
It uses 2 layers: morphological disambiguation and dependency annotation layers, sequentially.
In the dependency annotation layer, annotators are expected to click on a dependent and drag the cursor to its head.
It stores each sentence's annotation in a different file.
For backups of the annotations, it uses Dropbox~\cite{dropbox}, which is a proprietary cloud storage platform.
There were several treebanks annotated using this tool: 1 English and 5 Turkish treebanks.

BoAT-v2 is the second iteration of \boatvone.
It improves on it by including a search functionality, an auto-completion feature, a database to represent sentences in a more essential way rather than a plaintext file, multiple dependency graph visualization options, and a web interface with an API for flexibility and accessibility.
\boatvtwo\ has been developed to reduce the clutter that \boatvone\ was reported to have in parts of the screen in the feedback of the previous \boatvone\ usage.

% \newcolumntype{s}{>{\columncolor[HTML]{FFFAF0}} p{2.8cm}}
% FFFAF0 : floral white
\begin{table}[ht]
    \label{table:rel-work}
    \centering
    \begin{tabular}{|s|>{\centering\arraybackslash} p{1.1cm}|>{\centering\arraybackslash} p{1.6cm}|>{\centering\arraybackslash} p{2.4cm}|>{\centering\arraybackslash} p{1.9cm}|>{\centering\arraybackslash} p{1.2cm}|}
       \hline
        \rowcolor[HTML]{FFFAF0} \multicolumn{6}{|c|}{\textbf{Comparison of Annotation Tools}} \\ \hline\hline
        \multicolumn{1}{|c|}{\cellcolor[HTML]{FFFAF0} \textbf{Tool}} & \textbf{Mouse} & \textbf{Keyboard} & \textbf{Standards} & \textbf{Multi-user} & \textbf{Search} \\\hline
        brat & + & + & Extendable & + & + \\\hline
        UD Annotatrix & + & + & \ud\ \& CG3 & + & - \\\hline
        DgAnnotator & + & - & \ud\ & - & - \\\hline
        CoNLL-U Editor & + & + & \ud\ & + & + \\\hline
        ITU & + & + & \ud\ & - & - \\\hline
        WebAnno & + & + & \ud\ & + & - \\\hline
        StarDust & + & + & - & - & - \\\hline
        \boatvtwo\ & + & + & \ud\ & + & + \\\hline
    \end{tabular}
    \caption{Properties of the above-mentioned annotation tools, namely whether they support mouse or keyboard usage, what standards they abide by, and whether they support multi-annotator treebanks or a search functionality.}
\end{table}
