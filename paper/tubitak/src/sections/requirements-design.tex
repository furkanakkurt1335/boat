\section{Requirements and Design}
\label{sec:requirements}

% Requirements elicitation and validation meetings were held with annotators who worked on the \bountreebank\ involving thousands of sentences to identify the software requirements.
Requirements elicitation and validation meetings were held with annotators who worked on a treebank~\cite{anon} involving thousands of sentences to identify the software requirements.
The main requirements that emerged are:
\begin{itemize}[before=\normalfont, font=\itshape, align=left,noitemsep,topsep=0pt,parsep=3pt,partopsep=0pt,labelsep=3pt,align=left]
    \item[Support for sentence annotation:]
        To ensure speedy and accurate annotations, various features must be provided.
        In order to eliminate mistakes caused by typos while saving time for the annotator, \textit{auto-completion} should be supported whenever possible.
        Due to typologies of and syncretism being generally higher in MRLs, automatic parsing of such treebanks is difficult and they more often than not fail to create valid parsings of tokens.
        Thus, refining automatic parsing becomes an important task for agglutinative languages.
        Words in MRLs tend to have morphemes stacked on roots.
        Annotators must be able to refine or correct automatically parsed entries, which for agglutinative languages includes splitting or joining of lemmas.
    \item[Use of screen real estate and customization:]
        Annotators annotate many sentences in each session.
        For agglutinative languages, sentences tend to be long and complicated.
        The view of such sentences can cover a considerable part of the screen.
        The annotator must focus on the sentence being annotated, the lemmas, and all the features, which require lots of concentration.
        The annotation process must require little to no scrolling.
        For particularly long sentences, no scrolling may not be possible; however, the sentence being annotated should always be in sight of the annotator.
        Furthermore, every annotator has their unique preferences regarding how they would like to annotate.
        They must have some control on showing or hiding certain information according to their preferences.
        Overall, efficient use of screen real estate is important to provide the context needed to annotate long sentences.
    \item[Keyboard-oriented input:]
        Annotations of agglutinative languages require annotation of several morphological features for many tokens of a sentence in comparison to analytic languages like English.
        While drag-drop interfaces can work well for analytic languages, they don't for MRLs as they require more toggles between mouse and keyboard interactions which becomes frustrating.
        Keyboard-based interaction must be offered for all possible tasks so that the annotator can work without needing to switch between the keyboard and the mouse.
    \item[Collaborative annotation:]
        The large sizes of treebanks necessitate that they be annotated by multiple annotators.
        The tool must support multiple annotators working simultaneously on the same treebank.
        The ability to share annotations is vital for reference and consistency.
        Hence, the tool must provide a means for registering multiple annotators and keeping track of their respective annotations.
    \item[Search:]
        During the annotation of a sentence, annotators sometimes need to refer to previous annotations for guidance.
        An annotator should be able to search a treebank according to surface forms, and importantly, according to linguistic features.
        They should be able to perform complex searches that involve several criteria such as \ud\ tags, individual morphological features, and surface forms.
        This requirement reduces cognitive load and facilitates consistency among annotations.
\end{itemize}

Given these requirements, we decided that a web-based application that supports multiple authenticated users would be able to provide a collaborative platform for annotators.
The treebank and user annotations persists in a database which makes managing the data and searches much more reliable and efficient.
Also, an API is developed to support flexibility and extensibility.
Finally, in addition to making the tool open-source, we containerized the application using Docker~\cite{docker} to support its accessibility.
