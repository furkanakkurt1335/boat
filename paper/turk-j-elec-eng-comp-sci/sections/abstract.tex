\begin{abstract}
    The value of quality treebanks is steadily increasing due to the crucial role they play in the development of natural language processing tools.
    The creation of such treebanks is enormously labor-intensive and time-consuming.
    Especially when the size of treebanks is considered, tools that support the annotation process are essential.
    Various annotation tools have been proposed, however, they are often not suitable for agglutinative languages such as Turkish.
    \boatvone\ was developed for annotating dependency relations and was subsequently used to create the manually annotated \bountreebank\ (UD\_Turkish-BOUN).
    In this work, we report on the design and implementation of a dependency annotation tool (\boatvtwo) based on the experiences gained from the use of \boatvone, which revealed several opportunities for improvement.
    \boatvtwo\ is a multi-user and web-based dependency annotation tool that is designed with a focus on the annotator user experience to yield valid annotations.
    The main objectives of the tool are to: (1) support creating valid and consistent annotations with increased speed, (2) significantly improve the user experience of the annotator, (3) support collaboration among annotators, and (4) provide an open-source and easily deployable web-based annotation tool with a flexible application programming interface (API) to benefit the scientific community.
    This paper discusses the requirements elicitation, design, and implementation of \boatvtwo\ along with examples.

\keywords{natural language processing, linguistic annotation, annotation tool, web application, dependency parsing, Universal Dependencies}
\end{abstract}
