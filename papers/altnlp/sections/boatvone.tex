\section{BoAT-v1}
\label{sec:boatvone}

\boatvone~\cite{turk-etal-2019-turkish} is a standalone tool for annotating treebanks that is compatible with the \ud\ framework~\cite{UD} that is implemented using Python~\cite{python} and Qt~\cite{qt}.
It was specifically developed for annotating Turkish treebanks and is particularly suitable for agglutinative languages, however it can be used for other languages.
It supports the annotation of a single treebank at a time.
The annotations are stored in a file in \conllu\ format.
The file is updated during the annotation process.
It uses a validation script developed by \ud\ to display errors.
\boatvone\ was used to create the \bountreebank~\cite{turk2021resources,UD-Boun-Treebank} -- a manually annotated Turkish dependency treebank comprising close to 10 thousand sentences.

% main features of v1
For each sentence to be annotated, the annotator is shown a table which has a token per row with its corresponding tags (\id, \form, \lemma, \upos, \xpos, \feats, \head, \deprel, \deps, and \misc\ as detailed in~\cite{turk-etal-2019-turkish}).
The annotator manually enters values for each tag of each token.
It supports the splitting and joining of lemmas which is particularly significant for agglutinative languages.
Tokens that are split result in additional rows for each part.
Figure~\ref{fig:anno-fig-v1} in Appendix~\ref{sec:appendix-a} shows a sentence that is being annotated.
The token ``yoktu'' (ID: 4-5) is split into ``yok'' (ID: 4) and ``tu'' (ID: 5). 
Furthermore, it parses the \feats\ tag's value into individual morphological features.
The number of morphological features are much higher in agglutinative languages, therefore the value of the \feats\ tag often consists of numerous values.
As such, they are difficult to read.
\boatvone\ supports viewing these features individually under their associated features (Case=Nom|Number=Sing|Person=3 can be shown in columns ``Case'', ``Number'', and ``Person'' with the values ``Nom'', ``Sing'', and ``3'').
It also allows the annotators to be able to take notes for specific annotations.
