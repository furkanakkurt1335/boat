\section{Requirements and Design}
\label{sec:requirements}


Requirements elicitation and validation meetings were held with annotators who worked on the \bountreebank\ involving thousands of sentences to identify the software requirements.
The main requirements that emerged are:
\begin{itemize}[before=\normalfont, font=\itshape, align=left,noitemsep,topsep=0pt,parsep=3pt,partopsep=0pt,labelsep=3pt,align=left]
    \item[Collaborative annotation:]
        The sheer size of treebanks necessitates that they are annotated via multiple annotators.
        The tool must support multiple annotators working on the same treebank.
        The ability to share annotation experiences is vital for reference and consistency.
        Hence, the tool must provide means for registering multiple annotators and keeping track of their respective annotations.
    \item[Search:]
        While annotating a sentence, annotators may need to refer to previous annotations for guidance.
        An annotator should be able to search a treebank according to surface forms, and importantly, according to linguistic features.
        They should be able to perform complex searches that involve several criteria such as \ud\ tags, individual features of the \feats\ tag, and surface forms.
        This requirement reduces the cognitive load as well as supports consistency among annotations.
    \item[Keyboard-oriented input:]
        Annotations of agglutinative languages require annotation of several features for individual tokens in comparison to analytical languages like English.
        While drag-and-drop interfaces can work well for analytical languages, they don't for MRLs as they require more toggles between mouse and keyboard interactions which is frustrating.
        Keyboard-based interaction must be offered for all possible tasks.
    \item[Support for Sentence annotation:]
        To support the speed and accurate annotations, \textit{autocompletion} should be supported whenever possible.
        Due to typologies of and syncretism being generally higher in MRLs, automatic parsing of such treebanks is difficult and they more often than not fail to create valid parsings of tokens.
        Thus, refining automatic parsing is important for agglutinative languages, which requires splitting of lemmas.
        Words in MRLs tend to have morphemes stacked on roots.
        Annotators must be able to refine/correct automatically parsed entries, which for agglutinative languages includes \textit{splitting of lemmas}.
    \item[Use of screen real estate and customization:]
        Annotators annotate numerous sentences every session.
        For agglutinative languages, sentences tend to be long and complicated.
        The view of such sentences can cover a considerable part of the screen.
        The annotator must focus on the sentence being annotated, the lemmas, and all the features, which requires much concentration.
        The annotation process must not involve scrolling, if possible.
        For very long sentences, this may not be possible; however, the sentence being annotated should never go out of vision.
        Furthermore, each annotator has their unique preferences for how they annotate a sentence.
        They must have some control regarding showing or hiding certain information according to their preferences.
        Overall efficient use of the screen real estate is important to convey the context needed to annotate long sentences.
\end{itemize}

Given these requirements, we decided that a web-based application that supports multiple authenticated users would support a collaborative platform for annotators.
The treebank and user annotations will persist in a database which makes managing the data and searches much more reliable and efficient.
Also, an API is developed to support flexibility and extensibility.
Finally, in addition to making the tool open-source, we containerized the application using Docker\cite{docker} to support its accessibility.
