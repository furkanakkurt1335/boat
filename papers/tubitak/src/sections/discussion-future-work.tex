\section{Discussion and Conclusions}
\label{sec:discussion}

Our aim with \boatvtwo\ is to develop a collaborative web-based application based on previous experiences, supporting the annotation process of many languages, specifically morphologically rich ones.
We believe that having experts in linguistics and annotators experienced in agglutinative treebank creation was instrumental in understanding the requirements and also in the design process.
We held numerous elicitation interviews and further meetings for clarifications.

We have used modern software development tools and management practices during the development lifecycle of this tool.
The API enables various extensions of the tool and also programmatic access to the treebanks.
The containerization with Docker~\cite{docker} eases the deployment procedure.
% It is available on Boğaziçi University's NLP platform~\cite{tulap} as an open-source resource.
It is available~\cite{anon} as an open-source resource.
The wide availability of the tool is valuable for future use and developments.

We are encouraged by the evaluation performed on this tool and anticipate its extensions.
The feedback coming from the evaluation presents many ways that the tool can be improved.
For the tool's future, we plan on improving the current functionality, implementing annotation-centric new features, such as automated annotation of certain tags based on rules provided by annotators, or creating notes on annotations to communicate with other annotators through the application.
We also plan to implement support for various languages regarding auto-completion and annotation validation.
% annotation types outside dependency annotation task?
% mention that tool will be used soon by Busra
% conditional agreement
% tools to preprocess sentence data (NLP)