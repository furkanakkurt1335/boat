\section{Background}
\label{sec:background}

\subsection{BoAT-v1}
\label{sec:boatvone}

\boatvone~\cite{turk-2021-boat} is a standalone tool for annotating treebanks that are compatible with the \ud\ framework~\cite{UD}, implemented using Python~\cite{python} and Qt~\cite{qt}.
It was specifically developed for annotating Turkish treebanks and is particularly suitable for agglutinative languages; however, it can be used for other languages as well.
It supports annotation of a single treebank at a time.
Annotations are stored in a file in \conllu\ format.
The file is updated during the annotation process.
It uses a validation script developed by \ud\ to dynamically display errors to the annotator as they annotate.
\boatvone\ was used to create the \bountreebank~\cite{ud-boun-treebank} -- a manually annotated Turkish dependency treebank comprising 9,761 sentences from 5 different domains: essays, national newspapers, instructional texts, popular culture articles and biographical texts.

% main features of v1
For annotation of any sentence, an annotator is shown a table which has a token per row with its corresponding tags (\id, \form, \udlemma, \upos, \xpos, \feats, \head, \deprel, \deps, and \misc\ as detailed in~\cite{turk-2021-boat}).
The annotator manually enters values for each tag of each token.
It supports splitting and joining of lemmas which is particularly significant for agglutinative languages, since tokens in agglutinative languages are comprised of many affixes whose annotations require distinct lemmas.
Tokens that are split result in additional two rows whose lemmas are then manually adjusted by an annotator such that they become distinct parts of a token.
Figure~\ref{fig:anno-fig-v1} in Appendix~\ref{sec:appendix-a} shows annotation of a sentence.
The token ``yoktu'' (ID: 4-5) is split into ``yok'' (ID: 4) and ``tu'' (ID: 5). 
Furthermore, it parses the \feats\ tag's value into individual morphological features.
The number of morphological features are much higher in agglutinative languages, therefore the \feats\ tag often consists of numerous distinct features.
As such, they are difficult to read.
\boatvone\ supports viewing these features individually under their corresponding columns (e.g. Case=Nom|Number=Sing|Person=3 can be shown in columns ``Case'', ``Number'', and ``Person'' with the values ``Nom'', ``Sing'', and ``3'').
It also allows the annotators to be able to take notes for specific annotations.

% UD_English-ATIS/stats.xml:
% Non-unique feats: 63434
% Total tokens: 61879
% Feats per token: 1.0251296885857883

% UD_Turkish-ATIS/stats.xml:
% Non-unique feats: 112214
% Total tokens: 45875
% Feats per token: 2.4460817438692097

% UD_German-PUD/stats.xml:
% Non-unique feats: 56987
% Total tokens: 21000
% Feats per token: 2.7136666666666667

% UD_Turkish-PUD/stats.xml:
% Non-unique feats: 32583
% Total tokens: 16536
% Feats per token: 1.9704281567489115

% UD_Korean-PUD/stats.xml:
% Non-unique feats: 13430
% Total tokens: 16584
% Feats per token: 0.8098166907863

% UD_Spanish-PUD/stats.xml:
% Non-unique feats: 48819
% Total tokens: 22822
% Feats per token: 2.1391201472263606

% UD_Italian-PUD/stats.xml:
% Non-unique feats: 40715
% Total tokens: 22182
% Feats per token: 1.8354972500225408

% UD_French-PUD/stats.xml:
% Non-unique feats: 47560
% Total tokens: 24131
% Feats per token: 1.970908789523849

% UD_English-PUD/stats.xml:
% Non-unique feats: 23995
% Total tokens: 21176
% Feats per token: 1.1331224027200604

% explaining comparison in table

\newcolumntype{s}{>{\columncolor[HTML]{FFFAF0}} p{2.8cm}}
\begin{table}[h]
    \label{table:feat-comp}
    \centering
    \begin{tabular}{|s|>{\centering\arraybackslash} p{3.5cm}|>{\centering\arraybackslash} p{3cm}|>{\centering\arraybackslash} p{3cm}|}
       \hline
        \rowcolor[HTML]{FFFAF0} \multicolumn{4}{|c|}{\textbf{Comparison of UD Treebanks}} \\ \hline\hline
        \multicolumn{1}{|c|}{\cellcolor[HTML]{FFFAF0} \textbf{Treebank}} & \textbf{Non-unique feats} & \textbf{Total tokens} & \textbf{Feats per token} \\\hline
        UD\_Turkish-ATIS & 112214 & 45875 & 2.446 \\\hline
        UD\_English-ATIS & 63434 & 61879 & 1.025 \\\hline
        UD\_Turkish-PUD & 32583 & 16536 & 1.970 \\\hline
        UD\_English-PUD & 23995 & 21176 & 1.133 \\\hline
    \end{tabular}
    \caption{Comparison of morphological feature annotation in UD treebanks}
\end{table}
