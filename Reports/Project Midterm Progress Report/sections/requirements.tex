\chapter{SOFTWARE REQUIREMENTS}
\label{chapter:requirements}

This section presents the software requirements elicited and validated through interviews with an annotator who is responsible for the creation of the \bountreebank.

\section{Glossary for the Software Requirements Document}

\begin{itemize}[before=\normalfont, font=\itshape, align=left]

    \item[Annotation]: The task of parsing and identifying features of a text linguistically.

    \item[Autocomplete]: A shorthand writing system where a user can write the initials of a value and the system infers the value intended to be written.

    \item[Column]: A field of the forms chosen to be shown on the table by the annotator.

    \item[CoNLL-U]: A revised version of the CoNLL-X format.

    \item[Dependency]: Specifically syntactical dependencies such that one individual form depends upon another.

    \item[Feature]: A kind of information of a certain word. A list can be found here.

    \item[Field]: Parts of word lines. The possible fields are ID, FORM, LEMMA, UPOS, XPOS, FEATS, HEAD, DEPREL, DEPS, MISC and all the features individually. They are explained in detail here.

    \item[Form]: A part of a sentence, whether it be a word or a punctuation symbol.

    \item[Help Page]: A web page explaining how to use the tool and possible values for features.

    \item[Index]: A form's order in the sentence (also called Word Index) or a sentence's order in the conllu file.

    \item[Lemma]: The base form of a set of words (e.g. break for the set {break, breaks, breaking, broke}).

    \item[Note]: Text written for a specific sentence that's to be used to explain the annotation choices.

    \item[Phrase]: A word or a few words that make sense together that may not make sense by themselves individually.

    \item[Row]: A form or lemma shown together with its features on the table.

    \item[Sentence]: One word or several words put together that can be divided into its lemmas to be annotated.

    \item[Table]: A rectangle divided by rows and columns showing the sentence being annotated, and its fields and forms in tabular form.

    \item[Tree]: A tree-like graph showing the dependencies annotated by the user for the current sentence.

    \item[Universal Dependencies]: A framework for annotating grammars of various natural languages. Also known as UD.

    \item[User]: A linguist annotating sentences through the tool.

    \item[Word]: A string of characters allowed for a certain natural language with no spaces that makes sense together with other words in a sentence.

\end{itemize}

\section{Software Requirements Document (SRS)}

\markdownInput{sections/srs.md}

